\chapter{Argument}
%\fancyhead[LO]{Argument}

\paragraph{Contexte}
L’action se situe sur une arche transgénérationnelle lancée depuis la terre il y’a vingt-huit-mille ans depuis la Terre pour atteindre la planète Proxima centauri b trente-mille ans plus tard. Les liens avec la Terre ont été perdus depuis des milliers d’années et au sein de l’Arche des civilisations propres se sont développées. Au moment où se situe l’intrigue existent deux entités politiques que sont les royaumes antagonistes de \campprincipal{} et de \campoppose{}.
Les évènements de la pièce ont lieu essentiellement au sein de la cours royale de \campprincipal{} alors que la guerre avec \campoppose{} est sur le point d’éclater.

\paragraph{Acte I}
\elena{}, fils illégitime du roi du \campprincipal, dont la position au sein de l’armée et la popularité croit, pousse sa sœur \princesse{} à mener la prochaine bataille contre \campoppose{}. En fait, il accroit sa propre popularité afin que la reine \reine{}, mère de \princesse{}  s’efforce d’elle même à placer sa fille en tant que cheffe des armées.

Pendant que \reine{} et \princesse{} complotent contre \elena{},
ce dernier convainc le général \general{} de mener la bataille, accroissant ainsi l’ambition de ce dernier à concurrencer la princesse \princesse{}. Or, \elena{} souffle à \general{} des conseils qui le décrédibiliseront auprès du roi \roi{} lequel détient la haute main sur la décision à prendre.

Devant \roi{} qui doit prendre une décision, \elena{} simule l’incompétence afin d’être écarté, tandis que \general{} ne se montre pas suffisement à la hauteur. Le choix du roi se porte sur sa fille \princesse{} qui est par ailleurs son enfant préféré.

Avant la bataille, \elena{} envoie sous identité secrète le plan de bataille de \princesse{} à l’état-major de \campoppose{} ce pourquoi il reçoit un virement bancaire.

\paragraph{Acte II}
L’on apprend que la bataille s’est soldée par un échec cuisant pour le \campprincipal{} si bien que \princesse{} en est morte et qu’\roi{} en est inconsolable, au point que semble le gagner la folie. \elena{} en profite pour exciter l’ambition de \general{} en lui suggérant d’organiser un coup d’État. \general{} obtient même l’aval tacite du conseil des grands du royaume qui l’assure de le porter au pouvoir en cas de réussite.

Parallèlement, \elena{} vient en aide à son demi-frère \vladimir{}, lui aussi fils de \reine{}, victime d’un chantage de la part d’une des prostituées qu’il fréquente, \catin, et lui prête même la somme de trente-mille cosmis afin qu’il l’émette sur le compte bancaire de cette dernière.

À la fin de l’acte, \alexas{} précepteur et principal allié d’\elena{} lui apprends que l’expérience menée sous son contrôle sensée aboutir à une grossesse et une mise au monde naturelle a été couronné de succès, alors que depuis vingt-huit-mille ans les humains, pour des raisons techniques, ne pouvaient se reproduire que \xenism{in vitro}.


\paragraph{Acte III}
\elena{} et \alexas{} descendent dans la Cale, l’endroit le plus reclus de l’Arche qui tient lieu de cours des miracles, où \elena{} rencontre \ela{} sa sœur jumelle séparée à la naissance laquelle semble être devenue prostituée. Bien que demi-sœur de \princesse{}, \ela{} ressemble beaucoup à cette dernière. \elena{} profite de cette ressemblance pour manipuler \roi{} qui n’a rien à refuser à \ela{}. Ce dernier signe même un édit nommant \elena{} comme son successeur légitime.

\general{} envahi la salle du trône en compagnie de quelques soldats pour destituer \roi{} et s’aperçoit que celle qui semble être \princesse{} est bien vivante, contrairement à ce qu’il affirma au conseil seigneurial. Dans la confusion, un conjuré tue \general{}.

\paragraph{Acte IV}
\reine{} en colère contre \elena{} car le suspectant d’être à l’origine de la mort de sa fille et sachant que \ela{} n’est qu’un sosie de sa fille \princesse{} demande au parlement de diligenter un test d’ADN afin de statuer sur l’authenticité de la filiation entre \roi{} et \ela{}.

Bien que \ela{} ne fasse qu’usurper le rôle de \princesse{}, le test d’ADN la confirme comme fille d’\roi{} ce qui désappointe \reine. Dépitée, cette dernière tente de pousser son fils \vladimir{} peu porté sur la politique à convoiter le trône.

Pendant qu’\elena{} révèle à \ela{} son identité et son histoire, \vladimir{} est arrêté car ont été détecté sur ses comptes bancaires des mouvements d’argent équivalents à ceux que le \campoppose{} envoie au traitre qui leur a livré les plans de bataille ayant mené à la défaite.

\ela{} dont l’emprise sur \roi{} est plus grande que jamais le convainc de mettre \elena{} à la tête des armées pour la bataille à venir.

\paragraph{Acte V}
Tandis qu’une révolte populaire gronde dans la Cale, \elena{} parvient à la mater confirmant sa popularité au sein de l’armée.

À ce dernier échoit donc le privilège de la commander pour la bataille à venir. Bataille qu’il gagne totalement, car il a cette fois-ci livré de fausses informations à l’état-major de \campoppose{}. Le \campoppose{} tombe totalement, lui permettant de le faire annexer par \campprincipal{}.

\ela{}, qu’\roi{} au terme de sa folie pense toujours être \princesse{}, blâme ce dernier d’avoir permit la victoire à \elena{}, et le méprise si bien qu’il se suicide de chagrin.

Le triomphe d’\elena{} est total lors de la cérémonie de couronnement où est apporté le nourrisson né d’une grossesse naturelle, nommée \cleopatre{}. L’on apprends que cet enfant bien que porté par la mère porteuse \catin{} porte en fait les gènes d’\elena{} et \ela{}.

Lors du couronnement d’\elena{} et \ela{}, ces derniers sont plébiscités par le peuple, la noblesse, ainsi que l’ex-roi de \campoppose{} et sa suite. Les premières lois égalitaires sont prononcées par \elena{} tandis que sur le hublot central apparait Proxima centauri~b, réalisant ainsi une vieille prophétie.
%\lettrine[lines=3]{J}{e dois avouer} 

