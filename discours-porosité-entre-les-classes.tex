\chapter[Discours sur la porosité entre les classes inférieures]{Discours sur la porosité entre les classes inférieures\\du 2025 avril 15}
\fancyhead[LO]{Discours sur la porosité entre les classes inférieures}

\lettrine[lines=3]%
{D}{ans le tracé} que je donnais à la coupe transversale de la structure spatio-sociale de l’arche (\emph{Cf.} fig.\,\ref{fig:structure-spatio-sociale-arche} p.\,\pageref{fig:structure-spatio-sociale-arche}), je représentais assez spontanément sans trop y songer la frontière entre Cale et Sous-pont en pointillés pour la raison qu’au cours de l’écriture de la pièce, il me semblait que la Cale se devait être une espèce de lieu constitué sans être institué, revendiqué par aucun des \campprincipal{} ou \campoppose{}. Ce serait l’espèce de cours des miracles dont la frontières avec les cartiers des travailleurs du Sous-pont \incise{Que conservent jalousement chacun des deux royaumes} n’est pas très précise mais existe dans les esprits. Chacun la fait plus ou moins commencer à tel endroit ou tel autre, à mesure que l’environnement parait de plus en plus misérable ou dépasse un certain seuil critique d’insalubrité. Ce ventre mou de l’Arche sur lequel aucune puissance n’aurait de prétention territoriale, mais qui revêt une fonction cruciale pour les commandeurs du Pont, car ils y délèguent (d’aucun dirait aujourd’hui \autonym{soustraitent}) leurs basses œuvres, et tout ce avec quoi ils auraient honte d’admettre avoir partie liée, comme le proxénétisme. De cela, j’étais conscient, preuve en est que la circulation des personnages depuis le Sous-pont à la Cale ne cause pas plus de difficulté que, pour nous, d’emprunter la ligne 7 du bus. Là où le passage du Sous-pont au pont, décrit par une ligne contigüe sur la coupe transversale, nécessite de s’acquitter de blanc seing et saufconduit. Car le seul \emph{ruissellement} qui existe, est celui de l’urine des puissants qu’ils font tomber sur les faibles sans que jamais il ne remonte vers eux.

\paragraph{}
Si la raison microsociale  m’était claire pour les raisons qu’en exigeaient les péripétie, la raison macrosociale ne me parût que plus tardivement.
Car après réflexion, cette porosité entre le Sous-pont et la Cale, illustre
 illustre la théorie désignée par la locution \enquote{En prendre un pour taper sur l’autre} et menée par de nombreux mouvements politiques ou gouvernements afin de monter les faibles contre les encore-plus-faibles pour détourner le regard de la direction des dominants. C’est ainsi que les travailleurs moyens sont retournés contre les travailleurs pauvres, les travailleurs pauvres  contre les chômeurs, les chômeurs contre les étrangers, les étrangers  dit réguliers contre les étrangers dits clandestins.
 Compte aux étrangers clandestins, n’ayant personne à qui refiler l’anatème, n’aura pas d’avantage, pour paraphraser l’éternel Martin \bsc{Niemöller}, plus personne pour venir l’aider.

Car ceux qui auraient pu le faire sont trop occupés à  s’accuser les uns les autres de  de \enquote{manger le pain des \textins{nationaux}} comme l’aurait dit Fernand \bsc{Raynaud}. Sans qu’à aucun moment ne soit interrogé le rôle des bourgeois et des puissants dans leur précarité. Comme par hasard, jamais ceux qui ont capté le plus d’argent, de ressources, et de pouvoir économique ne sont désignés comme responsables, quoique ce soit ceux-là mêmes qui sont prompt à pointer du doigt tout un chacun en boucs-émissaire.

\paragraph{}
Or, la porosité de frontière entre les précaires et les encore-plus-précaires est une fonction charnière lorsqu’il s’agit d’en prendre l’un pour taper sur l’autre.
Il faut que le précaire sente que s’il ne tape pas l’encore-plus-précaire, il le deviendra. Telle est la menace qui lui est faite.

Et ce à travers la crainte constamment brandie du déclassement promis à tous. Mais pas à ceux d’en haut dont la frontière avec le bas est étanche. Car il faut que le glissement potentiel de chacun vers une situation plus basse soit ressenti comme probant et proche. Voilà pourquoi la frontière, au demeurant vague et implicite, entre Sous-pont et Cale est en pointillée. C’est car elle n’est pas un filet de sécurité mais seulement le constat émis à un moment donné que certains sont encore plus gueux que d’autres, sans que ces autres soient immunisés contre la gueuserie mise sous leur né tous les jours en guise de menace. Le seul moyen d’éviter de tomber dans le ravin encore plus profond de la misère, semble être pour ceux qui se sente à ses bords, de combler ce ravin-là par les encore-plus-précaires. Et voilà comment se conçoit le panier de crabes.

L’exemple illustrant le plus flagramment cette situation est celui du binôme entre les étrangers dit régulier et dit clandestin. L’étranger dit régulier, pétrifié par la surenchère de xénophobie érigée en concours par les éditocrates des organes de presse, s’empresse à vouloir jouer le bon élève. En cela, il pense pense, en désignant du doigt l’étranger dit clandestin, ne pas se voir coiffé du bonnet d’âne. Croyant bien naïvement par là s’arroger une isotélie. Ainsi qu’il en était fait par l’école de la Troisième République, avec le châtiment du symbole, où pour empêcher les langues régionales les locuteurs de se langues étaient invités à se livrer à une délation mutuelle.

\paragraph{}
Il y’a là une terrible ingénierie de la fourberie. Non contents de ne pas avoir à gérer le maintient des faibles et des encore-plus-faibles dans leur état, les premiers se chargeant de bien maintenir sous l’eau la tête des seconds, les puissants se gardent bien de se salir les mains et nous présentent la hauteur morale de ceux qui ne versent pas dans ce genre de bassesse.

Coup double ? Non, coup triple. Car par l’exemplarité de la hauteur morale qu’ils viennent de mettre en scène, ils justifient, s’il en était encore besoin, l’élection au rang où ils sont.

%Procédé particulièrement fourbe en ce qu’il s’apparente par là au châtiment du symbole pratiqué par l’école de la Troisième République pour que les locuteurs des langues qualifiées de régionales se châtient entre eux. Si efficace d’ailleurs qu’il fut réutilisé par le Royaume-Uni pour châtier les locuteurs du gaélique, par les États-Unis d’Amérique envers les primonatifs, et l’État du Japon pour réprimer l’okinawaïen.


%Car il faut que le travailleur moyen, tout juste nanti d’un mode de vie stable, sente qu’un licenciement toujours probable risque de le faire glisser vers un poste plus ingrat, précaire, ou moins valorisé. La secrétaire de direction doit craindre et sentir comme probante et \emph{proche} sa rétrogradation au rang d’ouvrière de ligne qui plie les cartons.
%
%\subparagraph{}
%L’ouvrier précaire à son tour doit son tire comme proche et probant son glissement vers le marasme, le loyer difficile à payer, et \xenism{in fine} sa relégation au rang de vagabond des rues. Mais plutôt que de désigner comme cause de cette relégation proche ou probable son loueur, son employeur qui le licencie, la vindicte sera redirigée vers ceux-là mêmes qui sont sa figure d’épouvantail que sont les vagabonds et sans-logi. 
%
%\subparagraph{}
%Ces derniers à leur tour, car il y’a là un implacable ruissellement, s’en prendront aux seuls auquels ils le peuvent que sont les étrangers.
%
%\subparagraph{}
%L’étranger dit régulier, pétrifié par la surenchère de xénophobie érigée en concours par les éditocrates des organes de presse, s’empresse à vouloir jouer le bon élève et pense, en désignant l’étranger dit clandestin, ne pas se voir coiffé du bonnet d’âne. Croyant bien naïvement par là s’arroger une isotélie.
%\subparagraph{}
%Et l’étranger clandestin, le plus souvent précaire, sans instruction ni connaissance des institutions et procédures, parfois même ne parlant pas même la langue, n’ayant pas de liaison sociale ou d’amis sur place, lui alors sur lequel se sont accumulées toutes les tares, lorsqu’il sera opprimé, expulsé parfois même vers des lieux où il sera maltraité ou même tué, lui alors, pour paraphraser l’éternel Martin \bsc{Niemöller},  n’aura plus personne pour venir l’aider. Car tous seront trop occupés à rejeter la balle sur autrui.
%
%\paragraph{}
%Mais dans toute cette chaine, ceux qui prennent la décision de licencier les autres, ceux qui les législateurs promulguent des lois restreignant l’accès des autres à la santé et au repos, abaissent leur salaires, ceux encore qui parmi les législateurs promulgueront les lois pour expulser les migrants, les immigrés, les réfugiés, les barbares, les métèques, les \xenism{gaikokujin}, \xenism{goyim}, ces personnages médiatiques qui à longueur de prêche télévisuel, colonnes de journaux, ou autre medium 
