
\chapter{Réminiscence du 2024 juin}
\fancyhead[LO]{Réminiscence du  2024 juin}

\lettrine[lines=3]{R}{étrospectivement, et même étrangement}
 longtemps après avoir finalisé cette pièce, je me rendit compte qu’un des facteurs ayant été grandement à l’œuvre dans l’inspiration et l’atmosphère générale qui en émanent, est une \autonym{œuvre} que je trouvais iconique, en l’occurrence le \work{Théâtre d’opéra spatial} généré à l’aide de \work{Midjourney} par l’opérateur de prompt Jason·Michèle \textsc{Allen} en 2022 septembre 5. Œuvre, s’il en est, à la portée particulièrement retentissante lors de sa publication, car grâce à elle l’opérateur de prompt reçu un prix lors d’un concours artistique en ayant omis de mentionner au jury qu’il l’avait réalisé\footnote{Ou devions-nous dire que \work{Midjourney} l’a réalisé ? La question est encore en suspens au sein de l’humanité à l’heure où j’écris ces lignes.} au travers d’une intelligence artificielle générative.


Je me souviens de la découverte que je fis de cette annonce dans mon agrégateur de flux de syndication. Certes, la nouvelle de cet événement où la production d’une machine remporta un prix artistique, annonçait ce qui semblait être l’ère nouvelle dans laquelle nous entrions ou nous croyons entrer. Mais au delà des conditions de production de l’œuvre ou du remous qu’elle suscitèrent les moyens mis en œuvre pour la réaliser, ce qui me frappa m’en souviens-je encore, est avant tout l’œuvre elle même et le sujet dont elle traite, car la la vision qu’elle projetait est iconique. Il y’avait quelque chose de grandiose à son travers, quelque chose de particulièrement royal, d’impérial, qui a trait au déstin d’un peuple tout entier. Elle représente une espèce de scène de théâtre, d’opéra, ou même de cours royale (d’aucun dirait que c’est là la même chose) à la solenité débordante ouvrant sur une espèce de gigantesque hublo parfaitement rond. Comme d’ailleurs les conditions de sa production semblent ouvrir, dans notre réalité, sur le monde nouveau qui s’annonce.
D’entre les trois personnages que l’on devine être des dignitaires à l’avant plant, ce qui semble être la princesse se tourne totalement, non, elle se précipite avec enthousiasme vers la promesse captivante de ce hublot où l’on croit percevoir, où l’on veut voir une colline verdoyante. Ainsi, tourne-t-elle le dos à un monde empli d’un carcan de conventions et, ose-t-on l’imaginer, de corruption.

Cette vue au sein du hublot jaillit comme une digue du mensonge universel qui vient de céder, un vaisseau transgénérationnel ayant atteint son point d’arrivée, quelque chose d’insoupçonné mais qui sonne pourtant comme une évidence, l’acmé d’une réalisation collective que vient accréditer la multitude d’autres personnages s’amassant plus proches du hublot comme encore plus intrigués ou plus avides d’adhérer à la promesse que déploie l’horizon lumineux du hublot. Une assemblée plénière de l’humanité réunie en ekklesia où va se décréter le sacerdoce universel.
Et pourtant, dans le bas coté de cet événement fondateur
se meut un détail auquel je n’avais pas prêté attention au premier visionnement mais qui parait aujourd’hui significatif, en l’occurrence l’espèce de drapé rouge sur le coté droit qui abrite soit un meuble, soit un humain, soit un humain voulant se cacher entre le meuble et le tissus. Or, cela confère l’irrésistible sensation d’y voir un cardinal qui intrigue en coulisse, et dont le drapé rouge est semblable à celui que l’on voit dans \work{L’Éminence grise} au titre si bien à propos.
%qui contraste avec la noirceur dorée de l’interieur, lequel semble être un vaisseau spatial ou un pallais ou même \incise{pourquoi pas} les deux à la fois.

Dans cette image semble s’être concentré l’avènement de quelque chose de nouveau, sans doute inquiétant mais pourtant optimiste. Voilà tout ce que je vis dans cette image, dans ce \work{Théâtre d’opéra spatial} qui est pourtant sorti des entrailles de silicone d’une machine non pensante. Et encore, recourir à l’image physiciste du silicone, c’est ignorer béatement que les processus ayant mené à son élaboration sont avant tout purement logique et mathématiques, la matérialité des machines n’étant qu’un support à l’exécution des opérations.


Et pourtant, alors que de part mon métier d’analyste programmeur je suis confronté aux innovations de l’intelligence artificielle constamment, j’avais réussi à oublier que ce \work{Théâtre d’opéra spatial} était à la source de mon inspiration. Ce n’est que tout récemment, alors que le hasard me confronta à cette image, que je repensais à mon propre \work{Hélios Skylab} et que je fis le lien avec ce tableau généré par \work{Midjourney} quelques mois à peine avant que je ne commence à rédiger cette pièce.
