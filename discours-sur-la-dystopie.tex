\chapter[Discours sur la dystopicité]{Discours sur la dystopicité\\du 2024 octobre 20}
\fancyhead[LO]{Discours sur la dystopicité}

\lettrine[lines=3]%
{I}{l est un mot que} je n’ai fort étrangement soulevé ni dans l’œuvre ni dans l’appareil que j’en ai donné, qui est celui de \autonym{dystopie}. Alors qu’évidement il s’agit là d’un sujet qui charpente totalement aussi bien la narration que l’esthétique, et dont de surcroit j’use jusqu’au galvaudage d’ordinaire.

\paragraph{Société}
{\em\normalsize Classes}~
Pourtant, tout s’y prête puisque l’\work{Ascension} traite d’anticipation sociale avec des États totalitaires organisant une violence n’épargnant aucune classe sociale, en plus de la stratification de celle-ci, au sein d’une situation écologique désastreuse. En particulier \incise{comment n’ai-je pu y penser que maintenant ?} les humains sont conditionnés en fonctions des couches sociales auxquelles ils sont dévolus avant même la naissance, tout autant que ceux du \citetitle{meilleurDesMondes}\footcite{meilleurDesMondes}. En suite, tout au long de leur existence, ils sont assignés à des cartiers au sein de l’aménagement de l’espace de l’Arche (\emph{Cf.} fig.\,\ref{fig:structure-spatio-sociale-arche} p.\,\pageref{fig:structure-spatio-sociale-arche}) autant que les trois classes de \citetitle{1984}\footcite{1984}, voir même aux fonctions tripartites indo-européennes ayant vu leur acmé sous la France d’ancien Régime. Comme l’Arche en tant que véhicule spatial est tridimensionnel, les zones les plus prisées sont les plus superficielles, ayant des hublots et permettant un accès plus aisé à l’espace, ainsi qu’il en est sur les bateaux de croisières où les chambres les plus chères sont celles ayant un hublot.

  \begin{table}[htbp]
    \centering
    \begin{tabular}{llll}\toprule
      Contexte                          & Supérieure      & Moyenne         & Inférieure \\\cmidrule(lr){2-4}
      %%%%%%%%%%%%%%%%%%%%%%%%%%%%%%%%%%%%%%%%%%%%%%%%%%%%%%%%%%%%%%%%%%%%
      Tripartition indo-européenne      & Oratores        & Bellatores      & Laboratores \\
      France d’Ancien Régime            & Clergé          & Noblesse        & Tiers-État \\
      \citetitle{1984}                  & Parti intérieur & Parti extérieur & Prolétaires \\
      \citetitle{meilleurDesMondes}     & Alpha           & Bêta et Gamma   & Delta et Epsilon \\
      \work{Ascension d’Hélios Skylab}  & Pont            & Souspont        & Cale \\ \bottomrule
    \end{tabular}
    \caption{Correspondance des classes sociales dans les différents contexte historique et œuvres dystopiques}
  \end{table}

  {\em\normalsize Aménagement spatial de l’Arche}~
  Évidement, on aurait tout aussi bien pu imaginer qu’au contraire, les zones les plus intérieures donc protégées des rayons cosmiques ou des collisions auraient pu être davantage prisées, il y’a là un choix à faire et j’ai jugé que narrativement la proximité avec l’organisation des bateaux de croisière filait davantage la métaphore navale. Et ce d’autant que s’il est des royaumes portant les noms des latéralisations navales, les classes sociales archiennes portent les noms de l’organisation des embarcation de sorte que le prestige de chacune de ces classes correspondent à la connotation plutôt méliorative du pont ou péjorative de la cale. Ce qui d’ailleurs engendre une éloquente assonance avec ceux de la Cale appelles \autonym{caleux} ce qui, à un \autonym{l} prêt, évoque les mains calleuses des ouvriers.

  \paragraph{Écologie}
  Or, l’aspect dystopique est accru par la principale préoccupation qu’est la finitude des ressources, certes résultant d’une pénurie organisée et ce faisant encore plus grave.
  Car cette Arche est à la fois la conséquence de l’avarie écologique ayant lieux sur la Vieille-Terre qu’une métaphore de celle-ci, l’histoire se répétant inlassablement. On aurait pu croire que les humains ayant prit acte de leurs actions passées jusqu’à la ressentir dans leur chair et leur habitat naturel se sont assagit, et l’on ferait preuve d’une touchante naïveté. Si à l’heure où j’ai encore le luxe et l’électricité d’écrire ces lignes l’humanité fonce vers le ravin, dans l’Arche elle est déjà en chute libre dans le gouffre et pourtant appuie sur l’accélérateur et klaxonne. Ou pour filer la métaphore maritime, elle est dans le maelström et met les voiles. En fait d’Arche, c’est disai-je davantage le radeau d’une humanité désespérante d’inhumanité. Un radeau de la Méduse répétè-je (\emph{Cf.} p.\,\pageref{par:radeaumeduse}), où l’anthropophagie sous le verni aseptisant de la technique s’est industrialisée à la manière de \citetitle{soleilVert}\footcite{soleilVert}. Car plus que jamais, au sens littéral, cette humanité est sur le même bateau mais continue pourtant à se diviser, à être tumultueuse sur un frêle esquif à l’équilibre instable. C’est la configuration clauswitzienne, l’équilibre de la terreur, où les conditions de l’explosion sont plus minces mais dont les conséquences ne s’en révéleraient que plus dévastatrices. Et pourtant, on trouve malgré tout le moyen de se diviser en deux royaumes, les dissension se faufilent au sein d’un même État entre classes sociales, et s’immissent jusqu’à la famille en rivalités de fratrie.

  \paragraph{Esthétique}
  Néanmoins, comme toute œuvre dystopique, si j’eu à traiter de l’anticipation sociale mais aussi technique, il n’y a pas ou pas de façon prégnante d’esthétique futuriste telle qu’engendrée par les artistes Italiens des années 1910 et 1920. Chacun des lecteurs ou metteurs en scènes s’imaginera d’après mon texte les costumes, et les décors qu’il voudra. Mais il ne m’a pas semblé utile que viennent à l’appuis du propos un imaginaire d’architecture brutaliste, d’acier brossé, ou de multiples lampes rouges jaunes bleues qui font bip et qui font flash. La technique est certes omniprésente dans l’œuvre, sans se réduire à un décors de fond elle a pleine intrication narrative. Mais il n’est pas utile pour être un biologiste s’adonnant à des manipulations \xenism{in vitro} de haut vol de s’habiller en combinaison moulante ne lycra de marvel ; en pourpoint, collerette, et trousse renaissantes il serait capable d’en faire tout autant.

  {\em\normalsize Science fiction sans futurisme}~
  Dans ce contexte de science-fiction, nul besoin de futurisme à l’appuis du propos. Des décors de bois, de fer forgé bien renaissants ou baroques, s’accommodent tout autant de la science. Ainsi qu’il en est dans \work{Dune} de Frank \textsc{Herbert}. À ce sujet d’ailleurs, il m’a semblé y’avoir une intéressante opposition sur un repère ayant pour abscisse la science fiction et pour ordonnée le futurisme, entre \work{Dune} et \work{La Guerre des étoiles} de George \textsc{Lucas}. Là où \work{Dune} est de la science fiction sans être futuriste, \work{La Guerre des étoiles} est futuriste sans être de la science fiction.

  {\em\normalsize Opposition entre \work{Dune} et \work{La Guerre des étoiles}}~
  En effet l’anticipation technique dans le second ne module en rien les institutions ni les relations sociales. \work{La Guerre des étoiles} aurait très bien pu être un roman de chevalerie où les races extraterrestres ne seraient que des peuples différents, les planètes des pays ou royaumes, et les sabres laser des sabres pas laser, que l’histoire n’en pâtirait nullement. Dans \work{La Guerre des étoiles}, le cortège technique n’est qu’un effet esthétique.
  Or, dans \work{Dune} où la triade religion—technique—politique domine, l’imbrication du deuxième terme avec deux autres est si importante qu’on ne peut envisager l’intrigue sans technique. Pourtant, dans les diverses adaptations cinématographiques ou télévisuelles qui en furent faites, le costume autant que l’architecture ne dépayseraient pas un homme de la renaissance.

  {\em\normalsize Cosmisme}~
  N’étant pas futuriste, l’œuvre n’en relève pas moins du cosmisme russe. Car est fait état d’un propos qui, comme dans \work{Dune}, fait que la marche de l’humanité suit une voie où le progrès sert sert des ambitions d’aucun jugerait spirituelles et prophétiques. Il y’a là une destinée manifeste.

  \paragraph{Connexité}
  Enfin, il s’est trouvé que fort coïncidemment (Je vous le jure !), durant la même année où j’eu fini l’écriture de ma pièce une œuvre télévisuelle américaine reprenait une idée à peu près semblable à celle développée ici mais dont des éléments présentes des similarités troublantes. Dans cette série, comme dans ma pièce, les protagonistes se trouvent dans une arche portant justement le nom de \citetitle{theArk}\footcite{theArk}. Dans les deux cas les deux se dirigent précisément vers Proxima centauri~b. Et enfin, dans les deux cas, les protagonistes sont dans la phase quasiment finale où ils approchent de leur objectif. 
  Une nuance toute fois, dans \citetitle{theArk}, le voyage est effectué en cent ans, quand dans la mienne il nécessite trente-mille ans (soit le terme le plus court selon les projections optimistes, les pessimistes diront septante-mille ans). C’était bien la peine que j’imagine tout un périple multi-millénaire avec des civilisations qui se seraient succédées et une situation géopolitique complexe, me suis-je dis.



